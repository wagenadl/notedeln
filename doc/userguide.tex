\documentclass[11pt]{report}
\usepackage{newcent}
\usepackage[letterpaper]{geometry}
\begin{document}
\thispagestyle{empty}
\begin{centering}
  {\Huge eln}
  \vskip30pt

  {\Large an Electronic Lab Notebook}
  \vskip60pt

  {\large By Daniel A. Wagenaar}
  \vfill
  
  {Copyright (c) 2013}
  
\end{centering}
\pagebreak
~
\vfill
\noindent Copyright (C) 2013 Daniel A. Wagenaar\medskip

``eln'' is free software: you can redistribute it and/or modify
it under the terms of the GNU General Public License as published by
the Free Software Foundation, either version 3 of the License, or
(at your option) any later version.

This program is distributed in the hope that it will be useful,
but WITHOUT ANY WARRANTY; without even the implied warranty of
MERCHANTABILITY or FITNESS FOR A PARTICULAR PURPOSE.  See the
GNU General Public License for more details.

You should have received a copy of the GNU General Public License
along with this program.  If not, see $<$http://www.gnu.org/licenses/$>$.
\pagebreak

\chapter{Introduction}

This document describes the installation and usage of ``eln'', an
electronic lab notebook written by Daniel Wagenaar. There are any
number of software packages available that implement electronic
notebooks. So why should you choose ``eln''? Eln is for you if:
\begin{itemize}
  \item You want your notes to be stored in a human-readable format.
  \item You want your notes to be stored in a format that will be easy to
    parse electronically even 500 years from now.
  \item You want your notes to be protected against accidental
    deletion.
  \item You want your notes to be automatically dated.
  \item You want to concentrate on entering text and not on
    formatting.
  \item You want to be able to include images and simple graphics with
    your notes and you want that to be easy.
  \item You want your notebook software to be fast, even with 100s
    of pages of notes.
\end{itemize}
\noindent Eln is not for you if:
\begin{itemize}
  \item You want complete fine control over the formatting of your notes
    (but eln will allow you some coarse control).
  \item You need to typeset complex equations in your notes (but eln
    will allow you to set basic equations).
  \item You need to typeset music in your notes.
  \item You need to import formatted documents into your notes (but
    eln can archive web pages and pdf files for you; they just cannot
    be rendered onto the notebook pages).
\end{itemize}

Eln notebooks consist of ``entries'' that fill one or more ``pages.''
Each entry has a title and consists of paragraphs of text and/or
graphics canvases.

This introduction
will not cover why you should keep a lab notebook, nor why an
electronic lab notebook may be desirable. You already know that. 

\chapter{Installation}

Installation on Windows should be easy using the provided ``eln.msi''
installation package. Installation on Debian, Ubuntu, or Mint Linux
should be equally easy using the provided ``eln.deb'' installation
package. At present, installation on other flavors of Linux will
require compiling the sources yourself. (Start from the provided
``eln.tar.gz'' archive or checkout the bzr source.) I do not own a
Mac, so I cannot provide a Mac OS version. But ``eln'' uses Qt for its
graphical user interface, so compiling it on a Mac should in principle
be straightforward. I would be happy to distribute a Mac installation
package of anyone is willing to make one.

\chapter{Using eln}


\section{Creating a new notebook}
\section{Creating new entries}
\section{Adding text}
\section{Adding graphics}
\section{Navigation}
\section{Editing old entries}
\section{Formatting and special characters}
\section{Exporting and printing}
\end{document}
